\documentclass{article}
\usepackage[brazil]{babel}
\usepackage{pdfx}
\usepackage{bookmark}
\usepackage{float}
\usepackage{svg}
\svgpath{{../data}}
\usepackage{listings}

\title{Implementação dos algoritmos {\it Dijkstra\/} e DFS sobre grafos}
\author{\sc{Paulo Ricardo \& Vinícios Bidin}}
\date{16 de maio de 2023}

\begin{document}

    \maketitle
	\tableofcontents

    \section{Introdução}
        \paragraph{} Trata-se de um projeto que tem como proposta a implementação dos algoritmos {\it Dijkstra} e DFS ou {(\it Depth First Search)} sobre grafos. Sendo o primeiro para busca do caminho mínimo entre dois vértices, especificamente para grafos ponderados, e o segundo uma busca em produndidade.
		\par O grafo em que estes algoritmos é aplicado se trata de um grafo pequeno com somente oito vértices que representam paradas de ônibus e as arestas com seus pesos, as distâncias entre esses.

		\section{Compilação e Execução}
			\subsection{Dependências}
				\paragraph{} São dependências do projeto, a biblioteca {\it Graphviz} sendo utilizada a linguagem {\it Dot} para a visualização do grafo gerado, utiliza-se ainda o {\it Makefile} para compilação de todo o projeto, para que não seja necessário ficar compilando os arquivos de forma individual.

			\subsection{Compilação}
				\paragraph{} Uma vez no diretório do projeto, compile, utilizando \texttt{make}.
				Outra maneira, é utilizando o comando \texttt{make run}. Desta forma, além de compilar, o binário \texttt{main} será executado, gerando os arquivos ``data/dfs.dot'' e ``data/dijkstra.dot''.
				Ainda, utilizando o comando \texttt{make plot}, além de compilado e executado, os vetores {\it (Scalable Vector Graphics)} ou {\it (svg)} serão gerados, seus caminhos serão ``data/dfs.svg" e ``data/dijkstra.svg".

			\subsection{Execução}
				\paragraph{} Para executar, basta executar o arquivo binário de saída \texttt{main}, utilizando o comando \texttt{./main}. Após feito, assim como rodando o comando \texttt{make run}, os arquivos {\it .dot} serão gerados, precisando então gerar os vetores a partir destes.

		\section{Resultados}
			\subsection{Dijkstra}
				\paragraph{} O algoritmo de {\it Dijkstra}, dado um grafo ponderado e os vértices de origem e destino, encontrará o melhor caminho tendo este, o menor custo possível. Isto é feito com o auxílio de uma lista de prioridades de vértices e suas distâncias ao vértice de origem. Inicialmente com exceção do vértice de origem que possui 0, todos os demais vértices são marcados com uma distância inifinita. Repeditamente, o algoritmo seleciona o vértice com a menor distância e verifica todos os vizinhos (vértices alcançáveis com um passo deste), e atualizando suas distâncias caso uma distância menor é encontrada. Este processo é feito até que o vértice de destino seja alcançado. Sua complexidade temporal pode ser limitada tanto inferior quanto superiormente pela função $\Theta$($|E| + |V|lg|V|$), sendo $|E|$ o número de arestas no grafo e $|V|$ o número de vértices.
				\par{} Segue a representação do grafo descrito feito com o auxílio do {\it Graphviz}, como mencionado anteriormente. Neste grafo, o caminho mínimo do vértice de origem {\it 0} ao vértice de destino {\it 7} está destacado.
				\begin{figure}[H]
					\centering
					\includesvg{dijkstra}
					\caption{Caminho mínimo de {\it Dijkstra}}
					\label{fig:dijkstra}
				\end{figure}

				\subsection{DFS}
					\paragraph{} A busca em profundidade (DFS) é um algoritmo de travessia de grafos que explora o máximo possível ao longo de cada ramificação antes de retroceder. Ele começa em um nó escolhido e visita todos os seus vizinhos, em seguida, recursivamente, explora os vizinhos ainda não visitados até que não os haja mais. Utiliza um vetor para manter o controle dos nós a serem visitados. Inicialmente, o nó de partida é inserido no vetor. Em cada passo, um nó é removido do vetor e marcado como visitado. Em seguida, todos os seus vizinhos não visitados são inseridos na vetor. Esse processo continua até que o vetor esteja vazio, indicando que todos os nós foram visitados. Para tal, a complexidade temporal pode ser definida por $O(|V| + |E|)$, onde $|V|$ é o número de vértices no grafo e $|E|$ o número de arestas deste.
					\par{} Na figura a seguir, tem-se representado o mesmo grafo apresentado anteriormente, porém desta vez sem peso nas arestas, e ainda, destacado da cor vermelha tem-se os vértices visitados para fazer uma busca em profundidade do nó {\it 0} até novamente o nó {\it 7}.
					\begin{figure}[H]
						\centering
						\includesvg{dfs}
						\caption{Busca em profundidade}
						\label{fig:DFS}
					\end{figure}

\end{document}
