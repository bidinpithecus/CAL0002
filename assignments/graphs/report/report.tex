\documentclass{article}

\usepackage{listings}

\title{Implementação dos algoritmos {\it Dijkstra\/} e DFS sobre grafos}
\author{\sc{Paulo Ricardo \& Vinícios Bidin}}
\date{4 de maio de 2023}

\begin{document}

    \maketitle

    \section{Introdução}
        \paragraph{} Trata-se de um projeto que tem como proposta a implementação dos algoritmos {\it Dijkstra} e DFS ou {(\it Depth First Search)} sobre grafos. Sendo o primeiro para busca do caminho mínimo entre um dois vértices, especificamente para grafos ponderados, e o segundo uma busca em produndidade.
		\par O grafo em que estes algoritmos é aplicado se trata de um grafo pequeno com oito vértices que representam paradas de ônibus e as arestas com seus pesos as possíveis distâncias entre esses.

		\section{Compilação}
			\subsection{Dependências}
				\paragraph{} São dependências do projeto, a biblioteca {\it Graphviz} sendo utilizada a linguagem {\it Dot} para a visualização do grafo gerado, utiliza-se ainda o {\it Makefile} para compilação de todo o projeto, para que não seja necessário ficar compilando os arquivos de forma individual.

			\subsection{Compilação}
				\paragraph{} Uma vez no diretório do projeto, compile, utilizando {\it make}.
				Outra maneira, é utilizando o comando {\it make run}. Desta forma, além de compilar, o binário {\it main} será executado, gerando os arquivos `data/dfs.dot` e `data/dijkstra.dot`.
				Ainda, utilizando o comando {\it make plot}, além de compilado e executado, os vetores {\it (Scalable Vector Graphics)} ou {\it (svg)} serão gerados, seus caminhos serão `data/dfs.svg` e `data/dijkstra.svg`.

			\subsection{Execução}
				\paragraph{} Para executar, basta executar o arquivo binário de saída {\it main}, utilizando o comando {\it ./main}. Após feito, assim como rodando o comando {\it make run}, os arquivos {\it .dot} serão gerados, precisando então gerar os vetores a partir destes.

		\section{Resultados}
			\paragraph{}

\end{document}
